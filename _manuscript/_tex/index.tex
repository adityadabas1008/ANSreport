% Options for packages loaded elsewhere
% Options for packages loaded elsewhere
\PassOptionsToPackage{unicode,linktoc=all}{hyperref}
\PassOptionsToPackage{hyphens}{url}
\PassOptionsToPackage{dvipsnames,svgnames,x11names}{xcolor}
%
\documentclass[
  ngerman,
  letterpaper,
  DIV=11]{scrreprt}
\usepackage{xcolor}
\usepackage[top=30mm,left=20mm,heightrounded]{geometry}
\usepackage{amsmath,amssymb}
\setcounter{secnumdepth}{5}
\usepackage{iftex}
\ifPDFTeX
  \usepackage[T1]{fontenc}
  \usepackage[utf8]{inputenc}
  \usepackage{textcomp} % provide euro and other symbols
\else % if luatex or xetex
  \usepackage{unicode-math} % this also loads fontspec
  \defaultfontfeatures{Scale=MatchLowercase}
  \defaultfontfeatures[\rmfamily]{Ligatures=TeX,Scale=1}
\fi
\usepackage{lmodern}
\ifPDFTeX\else
  % xetex/luatex font selection
  \setmainfont[]{Calibri}
\fi
% Use upquote if available, for straight quotes in verbatim environments
\IfFileExists{upquote.sty}{\usepackage{upquote}}{}
\IfFileExists{microtype.sty}{% use microtype if available
  \usepackage[]{microtype}
  \UseMicrotypeSet[protrusion]{basicmath} % disable protrusion for tt fonts
}{}
\makeatletter
\@ifundefined{KOMAClassName}{% if non-KOMA class
  \IfFileExists{parskip.sty}{%
    \usepackage{parskip}
  }{% else
    \setlength{\parindent}{0pt}
    \setlength{\parskip}{6pt plus 2pt minus 1pt}}
}{% if KOMA class
  \KOMAoptions{parskip=half}}
\makeatother
% Make \paragraph and \subparagraph free-standing
\makeatletter
\ifx\paragraph\undefined\else
  \let\oldparagraph\paragraph
  \renewcommand{\paragraph}{
    \@ifstar
      \xxxParagraphStar
      \xxxParagraphNoStar
  }
  \newcommand{\xxxParagraphStar}[1]{\oldparagraph*{#1}\mbox{}}
  \newcommand{\xxxParagraphNoStar}[1]{\oldparagraph{#1}\mbox{}}
\fi
\ifx\subparagraph\undefined\else
  \let\oldsubparagraph\subparagraph
  \renewcommand{\subparagraph}{
    \@ifstar
      \xxxSubParagraphStar
      \xxxSubParagraphNoStar
  }
  \newcommand{\xxxSubParagraphStar}[1]{\oldsubparagraph*{#1}\mbox{}}
  \newcommand{\xxxSubParagraphNoStar}[1]{\oldsubparagraph{#1}\mbox{}}
\fi
\makeatother


\usepackage{longtable,booktabs,array}
\usepackage{calc} % for calculating minipage widths
% Correct order of tables after \paragraph or \subparagraph
\usepackage{etoolbox}
\makeatletter
\patchcmd\longtable{\par}{\if@noskipsec\mbox{}\fi\par}{}{}
\makeatother
% Allow footnotes in longtable head/foot
\IfFileExists{footnotehyper.sty}{\usepackage{footnotehyper}}{\usepackage{footnote}}
\makesavenoteenv{longtable}
\usepackage{graphicx}
\makeatletter
\newsavebox\pandoc@box
\newcommand*\pandocbounded[1]{% scales image to fit in text height/width
  \sbox\pandoc@box{#1}%
  \Gscale@div\@tempa{\textheight}{\dimexpr\ht\pandoc@box+\dp\pandoc@box\relax}%
  \Gscale@div\@tempb{\linewidth}{\wd\pandoc@box}%
  \ifdim\@tempb\p@<\@tempa\p@\let\@tempa\@tempb\fi% select the smaller of both
  \ifdim\@tempa\p@<\p@\scalebox{\@tempa}{\usebox\pandoc@box}%
  \else\usebox{\pandoc@box}%
  \fi%
}
% Set default figure placement to htbp
\def\fps@figure{htbp}
\makeatother



\ifLuaTeX
\usepackage[bidi=basic]{babel}
\else
\usepackage[bidi=default]{babel}
\fi
\ifPDFTeX
\else
\babelfont{rm}[]{Calibri}
\fi
% get rid of language-specific shorthands (see #6817):
\let\LanguageShortHands\languageshorthands
\def\languageshorthands#1{}
\ifLuaTeX
  \usepackage[german]{selnolig} % disable illegal ligatures
\fi


\setlength{\emergencystretch}{3em} % prevent overfull lines

\providecommand{\tightlist}{%
  \setlength{\itemsep}{0pt}\setlength{\parskip}{0pt}}



 


\KOMAoption{captions}{tableheading}
\makeatletter
\@ifpackageloaded{caption}{}{\usepackage{caption}}
\AtBeginDocument{%
\ifdefined\contentsname
  \renewcommand*\contentsname{Inhaltsverzeichnis}
\else
  \newcommand\contentsname{Inhaltsverzeichnis}
\fi
\ifdefined\listfigurename
  \renewcommand*\listfigurename{Abbildungsverzeichnis}
\else
  \newcommand\listfigurename{Abbildungsverzeichnis}
\fi
\ifdefined\listtablename
  \renewcommand*\listtablename{Tabellenverzeichnis}
\else
  \newcommand\listtablename{Tabellenverzeichnis}
\fi
\ifdefined\figurename
  \renewcommand*\figurename{Abbildung}
\else
  \newcommand\figurename{Abbildung}
\fi
\ifdefined\tablename
  \renewcommand*\tablename{Tabelle}
\else
  \newcommand\tablename{Tabelle}
\fi
}
\@ifpackageloaded{float}{}{\usepackage{float}}
\floatstyle{ruled}
\@ifundefined{c@chapter}{\newfloat{codelisting}{h}{lop}}{\newfloat{codelisting}{h}{lop}[chapter]}
\floatname{codelisting}{Listing}
\newcommand*\listoflistings{\listof{codelisting}{Listingverzeichnis}}
\makeatother
\makeatletter
\makeatother
\makeatletter
\@ifpackageloaded{caption}{}{\usepackage{caption}}
\@ifpackageloaded{subcaption}{}{\usepackage{subcaption}}
\makeatother
\usepackage{bookmark}
\IfFileExists{xurl.sty}{\usepackage{xurl}}{} % add URL line breaks if available
\urlstyle{same}
\hypersetup{
  pdftitle={Entwurf und PCB Design eines universellen Biquad Filters},
  pdfauthor={Aditya Dabas (5230565); Mohamad Wehbi; Adrian Jauch},
  pdflang={de},
  colorlinks=true,
  linkcolor={blue},
  filecolor={Maroon},
  citecolor={Blue},
  urlcolor={Blue},
  pdfcreator={LaTeX via pandoc}}


\title{Entwurf und PCB Design eines universellen Biquad Filters}
\author{Aditya Dabas (5230565) \and Mohamad Wehbi \and Adrian Jauch}
\date{2025-06-29}
\begin{document}
\maketitle
\begin{abstract}
Lorem ipsum
\end{abstract}

\renewcommand*\contentsname{Inhaltsverzeichnis}
{
\hypersetup{linkcolor=}
\setcounter{tocdepth}{2}
\tableofcontents
}

\chapter{Einleitung}\label{einleitung}

Ziel dieses Versuchs ist es, ein universelles analoges Biquad-Filter zu
entwerfen, zu simulieren, praktisch aufzubauen und in ein
professionelles Leiterplattendesign zu überführen. Die Grundlage bildet
der Versuch 4 des ASLK PRO Manuals von Texas Instruments, in dem eine
zweistufige aktive Filterstruktur mithilfe von Operationsverstärkern
realisiert wird. Dabei wird das Verhalten typischer Filtertypen --
darunter Tiefpass, Hochpass, Bandpass und Bandsperre -- auf Basis einer
sogenannten Biquad-Schaltung untersucht.

Zu Beginn wird die im Manual auf Seite 32 dargestellte Schaltung in der
Simulationsumgebung KiCad modelliert, um ihr Frequenzverhalten zu
analysieren und theoretisch zu validieren. Anschließend wird dieselbe
Schaltung auf dem ASLK PRO Board unter Verwendung von Jumperkabeln
praktisch realisiert und vermessen. Durch den Vergleich der simulierten
und gemessenen Ergebnisse wird die Übereinstimmung zwischen Theorie und
Praxis überprüft.

Nach erfolgreicher Realisierung der Grundschaltung wird das Filter um
ein weiteres Modul ergänzt: Um einen Butterworth-Tiefpass dritter
Ordnung zu erhalten, wird das Ausgangssignal der Tiefpass-Stufe (LPF)
der Biquad-Schaltung genutzt, um einen zusätzlichen Integrator
anzusteuern. Diese Kaskadierung dient der Erzeugung eines glatten
Übergangs im Frequenzgang mit typischem Butterworth-Verhalten und stellt
eine praxisrelevante Anwendung dar.

Im dritten Schritt wird die gesamte Schaltung -- bestehend aus der
ursprünglichen Biquad-Konfiguration sowie dem optionalen
Butterworth-Modul -- in KiCad als Leiterplattenlayout umgesetzt. Dabei
wird darauf geachtet, die Filtercharakteristik durch gezielte Auswahl
von Jumperverbindungen flexibel messbar zu gestalten und sowohl die
Basis- als auch die Erweiterungsstruktur (Butterworth) in die Platine zu
integrieren. Nach erfolgreichem Design wird die Platine gefertigt,
bestückt und abschließend vermessen.

Durch diesen vollständigen Designzyklus wird ein praxisnahes Verständnis
für den analogen Schaltungsentwurf, die Simulation sowie die
Herausforderungen beim Layout und der Umsetzung eines funktionsfähigen
Leiterplattenprototyps vermittelt.

\chapter{Filter-Entwurf}\label{filter-entwurf}

\section{Bi-quad Filters}\label{bi-quad-filters}

Hier Bi-quad filters beschreiben (allgemein)

\subsection{Second Order Universal Active
Filter}\label{second-order-universal-active-filter}

Hier dann Bi-quad filter schaltung aus ASLK Pro Experiment-4

\section{Herleitung der Übertragungsfunktion bes Biquad
Filters}\label{herleitung-der-uxfcbertragungsfunktion-bes-biquad-filters}

Hier die Herleitung der Übertragungsfunktionen schreiben

\section{Filter Spezifikation und
Aufbau}\label{filter-spezifikation-und-aufbau}

Hier müssen wir die filter speizifikation aus Experiment-4 schreiben.

\subsection{Simulation der Schaltung}\label{simulation-der-schaltung}

Hier zeigen wir die Simulation von der Schaltung und die Plots

\subsection{Entuwrf der Schaltung auf ASLK
Lab-Kit}\label{entuwrf-der-schaltung-auf-aslk-lab-kit}

Aufbau von der Schaltung auf ASLK Lab-Board mit jumper Wires

\chapter{Charakterisierung}\label{charakterisierung}

\chapter{Messauswertung}\label{messauswertung}

\section{Messungen und Darstellung}\label{messungen-und-darstellung}

Hier schreiben wir das Vorgehen und was wir gemessen haben in der
diskrete Schaltung auf ASLK Lab-board und die Plots dazu

\section{Vergleich Simulation
vs.~Diskret}\label{vergleich-simulation-vs.-diskret}

hier machen wir ein vergleich zwischen Simulation und die auf dem Board
aufgebaute Schaltung mit Plots

\section{Diskussion der Messwerte}\label{diskussion-der-messwerte}

Hier erklären wir die Unterschiede zwischen Simulation und echte
Messwerte

\chapter{Entwurf des PCB-Designs}\label{entwurf-des-pcb-designs}

\section{Vorgehen}\label{vorgehen}

hier beschreiben wir das Vorgehen

\section{Messung der PCB-Schaltung}\label{messung-der-pcb-schaltung}

hier zeigen wir Plots von der PCB-Schaltung

\section{Diskret vs.~Integriert}\label{diskret-vs.-integriert}

hier vergleichen wir zwischen der Schaltung auf dem ASLK-Board und der
auf dem PCB

\chapter{Fazit}\label{fazit}

Hier schreiben wir Fazit von dem Experiment

\chapter*{Literaturverzeichnis}\label{literaturverzeichnis}
\addcontentsline{toc}{chapter}{Literaturverzeichnis}

\phantomsection\label{refs}




\end{document}
